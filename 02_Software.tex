\section{Software}

\subsection{Beispiele von Software}
\begin{itemize}
    \item Quellprogramme
    \item Bibliotheken
    \item Testprogramme
    \item Initialisierungsdaten
    \item Dialogtexte
    \item Anforderungsdokumentation
\end{itemize}

\subsection{System- vs. Softwareentwicklung}
\definition[0]{System}{Ein System ist aus Teilen (Systemkomponenten / Subsystemen; gegenständlich oder konzeptionell) zusammengesetzt, die untereinander in verschiedenen Beziehungen stehen und wechselwirken können.}
\definition[0]{Softwareentwicklung}{ist ausschließliche Entwicklung von Software.}
\definition[0]{Systementwicklung}{ist die Entwicklung eines Systems, das aus Hard- und Softwarekomponenten besteht. Bei deren Entwicklung auch Randbedingungen berücksichtigt werden müssen.}

\subsection{Änderung an Software der letzten Jahren}
\begin{itemize}
    \item Steigende Komplexität 
    \item Wachsender Anteil auf Mobilgeräten
    \item Vernetzung
    \item Steigende Qualitätsanforderungen
    \item Zunehmend mehr Altlasten und Außer-Haus-Entwicklung
\end{itemize}
Trotz der Tatsachen, dass Software an sich keinem Verschleiß unterliegt, ist sie nicht einfacher zu warten wie Apparaturen ähnlicher Komplexität, da Software vielen Abhängigkeiten unterliegt.