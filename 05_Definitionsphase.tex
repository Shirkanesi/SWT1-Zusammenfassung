\section{Definitions- / Analysephase}
\definition[0]{Definitionsphase}{In der Definitionsphase entsteht das \important{Pflichtenheft}}

\definition[0]{Pflichtenheft}{Das Pflichtenheft definiert(„modelliert“) das zu erstellende System (oder die Änderungen an einem existierenden System) so vollständig und exakt, dass Entwickler das System implementieren können, ohne nachfragen oder raten zu müssen, was zu implementieren ist.}
\imp[0]{}{Das Pflichtenheft beschreibt \important{nicht wie}, sondern nur \textbf{was} zu implementieren ist.}
\anmerkung[0]{}{Das Pflichtenheft ist eine Verfeinerung des Lastenhefts. In der Sprache der Entwickler.}

\subsection{Modellierung}
Das Pflichtenheft liefert ein Modell des zu implementierenden Systems.\\
Modell-Arten:
\begin{itemize}
    \item \important{Funktionales} Modell (aus dem Lastenheft)
    \begin{itemize}
        \item Szenarien und Anwendungsfall-Diagramme
    \end{itemize}
    \item \important{Objektmodell}
    \begin{itemize}
        \item Klassen- und Objektdiagramme
    \end{itemize}
    \item \important{Dynamisches} Modell
    \begin{itemize}
        \item Sequenzdiagramme
        \item Zustandsdiagramme
        \item Aktivitätsdiagramme
    \end{itemize}
\end{itemize}

\subsection{Realitäten eines Software-Ingenieurs}
Software-Ingenieure können verschiedene „Realitäten“ modellieren und realisieren:
\begin{itemize}
    \item Ein existierendes System (physikalisch, technisch, sozial oder Softwaresystem) modellieren und eine Realisierung bauen
    \begin{itemize}
        \item Das „Softwaresystem“ stellt einen wichtigen Spezialfall dar: Wir sprechen von Altlasten (engl. „Legacy-System“)
    \end{itemize}
    \item Eine Idee ohne entsprechendes Gegenstück in der Realität modellieren und realisieren
    \begin{itemize}
        \item Ein visionäres Szenario oder eine Kunden-Anforderung
        \item In solchen Fällen wird häufig nur ein Teil des Originalmodells gebaut, weil der Rest z.B. zu kompliziert, zu teuer oder unnütz ist
    \end{itemize}
\end{itemize}