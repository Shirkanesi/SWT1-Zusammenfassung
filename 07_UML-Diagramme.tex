\section{UML-Diagramme}

\subsection{Anwendungsfalldiagramm}
\begin{itemize}
    \item Zur \babyblue{Anforderungsspezifikation} – was will der Benutzer von seinem System?
    \item Modellieren typischer \babyblue{Interaktionen} des Benutzers mit dem System
    \item Ermöglicht Kontrolle, ob das System das vom Auftraggeber gewünschte leistet (Design und Implementierung)
\end{itemize}

\definition{Anwendungsfall}{Ein Anwendungsfall (engl. use-case) ist eine typische, gewollte Interaktion eines oder mehrerer Akteure (engl. actor) mit einem (geschäftlichen oder technischen) System.}

\fig{Beispiel „Groupware-System“}{./data/Anwendungsfalldiagramm.png}

\subsection{Aktivitätsdiagramm}
Beschreibe parallele und sequenzielle Abläufe
\begin{itemize}
    \item beschreibt einen Ablauf
    \item bestehen aus
    \begin{itemize}
        \item Aktions-, Objekt- und Kontrollknoten, sowie
        \item Objekt- und Kontrollflüssen
    \end{itemize}
\end{itemize}

\newpage

\anmerkung[0]{Beispiele von UML Diagrammen}{}

\fig{Beispiel Aktivitätsfalldiagramm}{./data/uml/Aktivitätsdiagramm.png}\\
\fig{Elemente eines Aktivitätsdiagramms}{./data/uml/elemente.png}\\   %Eventuell richtig machen idk lol.
\fig{Semantisches Beispiel}{./data/uml/aktivität_beispiel.png}\\
\fig{Synchronisieren}{./data/uml/aktivität_sync.png}

\newpage

\subsection{Interaktionsdiagramme}

\begin{itemize}
    \item zeigt die für einen bestimmten Zweck notwendigen Interaktionen zwischen Objekten
    \item Klassendiagramm ist Grundlage der Interaktionsdiagramme
    \item es existieren vier Typen:
    \begin{itemize}
        \item Kollaborationsdiagramm / Kommunikationsdiagramm
        \begin{itemize}
            \item Schwerpunkt: Struktur der Interaktionspartner
        \end{itemize}
        \item Zeitdiagramm
        \begin{itemize}
            \item Schwerpunkt: Zeitliche Koordination
        \end{itemize}
        \item Interaktionsübersicht
        \begin{itemize}
            \item Aktivitätsdiagramm zur Veranschaulichung komplexer Sequenzdiagramme
        \end{itemize}
        \item Sequenzdiagramm
        \begin{itemize}
            \item Schwerpunkt: Nachrichtenaustausch
        \end{itemize}
    \end{itemize}
\end{itemize}

\subsection{Sequenzdiagramm}
\fig{Sequenzdiagramm: Beispiel}{./data/uml/sequenz.png}

\begin{table}[h]
\begin{tabular}{l|l|l}
    Operator & Bed./Parameter                         & Bedeutung                                                                                \\
    \hline
    alt      & {[}bed.1{]},{[}bed.1{]},...,{[}else{]} & Nur eine der Alternativen wird ausgeführt                                                \\
    break    & {[}bed{]}                              & Ist die Bed. wahr, dann wir nur der Block ausgeführt und anschließend\\&& endet das Szenario. \\
    opt      & {[}bed{]}                              & Optionale Sequenz. Wird nur Ausgeführt, wenn Bed. wahr ist.                              \\
    par      &                                        & Enthaltene Teilsequenzen werden parallel ausgeführt.                                     \\
    loop     & {[}bed{]}                              & Solange die Bed. wahr ist, wird der Block ausgeführt.                                   
\end{tabular}
\caption{Operatoren zur Ablaufsteuerung}
\label{tab:ablaufsteuerung}
\end{table}

\subsection{Zustandsdiagramm}
Ein Zustandsübergang innerhalb eines einzelnen Objektes
\begin{itemize}
    \item wird durch ein Ereignis ausgelöst
    \item Übergang findet nur statt, wenn Übergangsereignis eintritt (guarded transition)
    \item $\epsilon$-Übergang braucht kein Ereignis sondern kann jederzeit erfolgen, wenn
    \begin{itemize}
        \item Das System sich in dem Zustand befindet und
        \item Die Bedingung erfüllt ist.
    \end{itemize}
    \item Spezielle Ereignisse
    \begin{itemize}
        \item at(ausdruck)
        \begin{itemize}
            \item Der Ausdruck beschreibt einen \babyblue{exakten absoluten Zeitpunkt}.
            \item bspl: at(7:30) - um 7:30 Uhr (überhaupt \important{nicht} früh)
        \end{itemize}
        \item after(audruck)
        \begin{itemize}
            \item Hier muss der Ausdruck einen \babyblue{relativen Zeitpunkt} beschreiben
            \item after(10min) - nach 10 min
        \end{itemize}
    \end{itemize}
\end{itemize}

\subsubsection{Aktionen}
\begin{itemize}
    \item Aktionen
    \begin{itemize}
        \item Mit einem Zustandsübergang kann eine Aktion verbunden sein.
        \item \babyblue{Eintrittsaktion} (entry action): wird beim Übergang in den Zustand ausgeführt. \\
        Schreibweise: entry / aktion()
        \item \babyblue{Fortlaufende Aktion}(do action): wird solange ausgeführt, wie in dem Zustand verweilt wird. \\
        Schreibweise: do / aktion()
        \item \babyblue{Austrittsaktion}(exit action): wird beim Übergang aus dem Zustand in einen anderen ausgeführt. \\
        Schreibweise: exiSt / aktion()
    \end{itemize}
    \item Eine Aktion wird sofort ausgeführt und benötigt keine (bzw. vernachlässigbare) Zeit
\end{itemize}

\subsubsection{Hierarchie}
\fig{Hierarchischer Zustandsautomat}{./data/uml/ZustandHirar.png}

\subsubsection{Nebenläuftigkeit}
Während System im Zustand G verweilt, kann es alle Zustandskombinationen aus E×F annehmen
\fig{Nebenläufige Automaten}{./data/uml/ZustandNebenlauf.png}

\subsection{Paketdiagramm}
\begin{itemize}
    \item Pakete sind Ansammlungen von Modellelementen (ME) beliebigen Typs (z.B. Anwendungsfälle, Klassen, …)
    \item Dient der Gliederung in überschaubare Einzeiten
    \item Abhängigkeit zwischen Paketen werden mit einem gestrichelten Pfeil dargesetellt.
\end{itemize}

\subsubsection{Modellelement}
\begin{itemize}
    \item Besitzt innerhalb eines Pakets einen eindeutigen Namen
    \item Kann mit Sichtbarkeit versehen werden
    \item Kann in anderen Paketen über seinen qualifiziereten Namen zitiert werden (Paketname::ME-Name)
\end{itemize}

\fig{Paketdiagramm Beispiel}{./data/uml/paket_beispiel.png}

\subsection{Sichtbarkeiten in Java}
\fig{Sichtbarkeiten in Java}{./data/javasichtbarkeit}
% Taken from: https://stackoverflow.com/questions/215497/what-is-the-difference-between-public-protected-package-private-and-private-in

\newpage
\section{Syntaktische Analyse}
\begin{table}[h]
\begin{tabular}{l|l|l}
Wortart            & Modellelement            & Beispiel                                 \\
\hline
Nomen              & Klasse                   & Auto, Hund                               \\
Namen              & Exemplar                 & Julian, Niklas, Möwe                     \\
Intransitives Verb & Methode                  & sitzen, atmen, laufen                    \\
Transitives Verb   & Assoziation oder Methode & abhängig von ... etw. essen, jmd. lieben \\
Verb ``sein''      & Vererbung/Exemplar       & ist eine (Art von)...                    \\
Verb ``haben''     & Aggregation              & hat ein, besteht aus                     \\
Modalverb          & Zusicherung              & müssen, sollen                           \\
Adjektiv           & Attribut                 & 3 Jahre alt               
\end{tabular}
\caption{Operatoren zur Ablaufsteuerung}
\label{tab:ablaufsteuerung2}
\end{table}

\subsection{Wann liegt (wahrscheinlich) keine Klasse vor}
\begin{itemize}
    \item Es lassen sich weder Attribute noch Operationen Identifizieren
    \item Eine Klasse enthält die gleichen Attribute und Methoden wie eine andere.
    \item Eine Klasse enthält nur Operationen, die sich anderen zuordnen lassen.
    \item Eine Klasse modelliert Implementierungsdetails
    \item Besitzt eine Klasse nur ein einziges oder wenige Attribute, so ist zu prüfen, ob diese Attribute nicht einer anderen Klasse zugeordnet werden könnten.
\end{itemize}

\subsection{Zulässige Kardinalitäten}
\begin{itemize}
    \item Muss-Beziehung
    \begin{itemize}
        \item Sobald das Objekt erzeugt ist, muss auch die Beziehung zu einem anderen Objekt aufgebaut werden.
    \end{itemize}
    \item Kann-Beziehung
    \begin{itemize}
        \item Die Beziehung kann zu einem beliebigen Zeitpunkt nach dem Erzeugen des Objekts aufgebaut werden.
    \end{itemize}
    \item Obergrenze fest oder variabel
    \begin{itemize}
        \item Ist eine Obergrenze von Problem her zwingend vorgegeben?
        \item Im Zweifelsfall mit variablen Obergrenzen arbeiten.
    \end{itemize}
\end{itemize}

Mache eine Klasse B erst zur Unterklasse einer Klasse A, wenn gezeigt werden kann, dass jedes Exemplar von B auch als ein Exemplar von A gesehen werden kann.