\section{Architekturstile}
\definition[0]{Architekturstil}{Die Architekturstile legen den \important{Grobaufbau} eines Softwaresystems fest.}

\subsection{Schichtenarchitektur}

\definition[0]{Schichtenarchitektur}{ist die Gliederung einer Softwarearchitektur in hierarchisch geordnete Schichten. Eine Schicht besteht aus einer Menge von Software- Komponenten (Module, Klassen, Objekte, Pakete) mit einer wohldefinierten Schnittstelle, nutzt die darunter liegenden Schichten und stellt seine Dienste darüber liegenden Schichten zur Verfügung.}
\anmerkung[1]{}{Zwischen den einzelnen Schichten ist die Benutztrelation linear, baumartig, oder ein azyklischer Graph. Innerhalb einer Schicht ist die Benutztrelation beliebig}
\fig{Beispiel für eine Drei-Schichten-Architektur}{./data/architekturstile/3_schichten_architektur.png}

\begin{itemize}
    \item Eine Schicht (engl. \textit{layer, tier}) ist ein Subsystem, welches Dienste für andere Schichten zur Verfügung stellt, mit folgenden Einschränkungen:
    \begin{itemize}
        \item Eine Schicht nutzt nur Dienste von niedrigeren Schichten
        \item Eine Schicht nutzt keine höheren Schichten
    \end{itemize}
    \item Eine Schicht kann horizontal in mehrere, unabhängige Subsysteme, auch Partitionen genannt, aufgeteilt werden
    \begin{itemize}
        \item Partitionen bieten Dienste für andere Partitionen der gleichen Schicht an
    \end{itemize}
\end{itemize}
\anmerkung[1]{Transparente und Intransparente Schichten}{Eine Schicht einer transparenten Schichtenarchitektur kann auf alle beliebigen unteren Schichten der Hierarchie direkt zugreifen. Schichten im intransparenten Modell müssen jede Schicht bis zur Zielschicht durchlaufen.}

\newpage

\subsection{Klient/Dienstgeber}

\definition[0]{Klient/Dienstgeber}{Ein oder mehrere Dienstgeber bieten Dienste für andere Subsysteme, Klienten genannt, an. Jeder Klient ruft eine Funktion des Dienstgebers auf, welcher den gewünschten Dienst ausführt und das Ergebnis zurückliefert. Jeder Klient muss hierzu die Schnittstelle des Dienstgebers kennen jedoch nicht jeder Dienstgeber die Schnittstelle aller Klienten.}
\begin{itemize}
    \item Ein Beispiel einer 2-stufigen, verteilten Architektur
    \item Wird oft beim Entwurf von Datenbank-Systemen verwendet:
    \begin{itemize}
        \item Front-End: Benutzeroberfläche für den Benutzer (Klient)
        \item Back-End: Datenbankzugriff und Manipulation (Dienstgeber)
    \end{itemize}
    \item Funktionen, die der Klient ausführt:
    \begin{itemize}
        \item Eingaben des Benutzers entgegennehmen
        \item Vorverarbeitung der Eingaben
    \end{itemize}
    \item Funktionen, die der Dienstgeber ausführt:
    \begin{itemize}
        \item Datenverwaltung
        \item Datenintegrität und -Konsistenz
        \item Sicherheit
    \end{itemize}
    \item Klient und Dienstgeber laufen i.d.R. auf unterschiedlichen Rechnern, aber nicht unbedingt.
\end{itemize}

Beispiel

\begin{itemize}
    \item Dateiübertragung auf einen FTP-Server:
    \begin{itemize}
        \item Der Klient (bspw. ein FTP-Programm wie Filezilla) initiiert das Übertragen einer Datei.
        \item Der Dienstgeber reagiert auf die Anfrage des Klienten und empfängt bzw. sendet die Datei.
    \end{itemize}
\end{itemize}
\newpage
\subsection{Partnernetze}

\definition[0]{Partnernetze}{Partnernetze sind Peer-to-Peer-Netze und eine Verallgemeinerung des Klient/Dienstgeber-Stils. Im Netz sind alle Teilnehmer gleichberechtigt und laufen auf unterschiedlichen Rechnern. Ein Teilnehmer ist gleichzeitig Klient und Dienstgeber.}
Eigenschaften:
\begin{itemize}
    \item Keine zentrale Koordination, keine zentrale Datenbasis. Jeder Partner kennt i.d.R. nur seine Nachbarpartner
    \item Gesamtverhalten wird durch Interaktion zwischen den Komponenten bestimmt
    \item Partner verhalten sich autonom
    \item Partner sind unzuverlässig (nicht immer an etc...)
    \item Daten müssen redundant verfügbar sein
\end{itemize}

\fig{Partnernetzstilstruktur}{./data/architekturstile/peertopeer.png}

\subsection{Datenablage}

\definition[0]{Datenablage}{Es gibt eine zentrale Ablagestelle für Daten während die Subsysteme lose gekoppelt nur darüber interagieren}

\fig{Datenablagenstilstruktur}{./data/architekturstile/depot.png}
\newpage
\subsection{Modell-Präsentation-Steuerung}

\definition[0]{Modell-Präsentation-Steuerung}{Die Modell-Präsentation-Steuerung (engl. Model-View-Controller) trennt Daten von deren Darstellung.}
\anmerkung[0]{(Daten-) Modell}{Das Subsystem zur Datenhaltung, welches neben Daten auch oftmals die Anwendungslogik beinhaltet.}
\anmerkung[0]{View (Präsentation oder Sicht)}{Verantwortlich für die Darstellung der Objekte der Anwendung.}
\anmerkung[0]{Controller (Steuerung)}{Das Subsystem für die Entgegennahme der Benutzereingaben und Steuerung der Interaktion zwischen Modell und Präsentation. Es aktualisiert das Modell und stößt ferner das melden von Änderungen der Modelldaten an die Präsentationen an. Die Steuerung kann zusätzlich einen Teil der Anwendungslogik enthalten.}



\subsection{Fließband}

\definition[0]{Fließband}{Jede Stufe des Fließbands ist eine eigenständig ablaufender Prozess oder Faden mit eigenem Befehlszähler. Stufen sind ggfs. mit Puffern verbunden, um Geschwindigkeitsdifferenzen auszugleichen}
\\
\fig{Fließbandstilstruktur}{./data/architekturstile/fließband.png}
\\
Stufen können bei Parallelrechnern echt paralell ablaufen (Pipelining) und damit beschleunigt werden. Bei sequentiellen Rechnern werden einzelne Prozesse abwechselnd ausgeführt.
\\
\fig{Fließbandverarbeitungsprinzip}{./data/architekturstile/fliesbandverarbeitung.png}
\\
Geignet für Verarbeitung von Datenströmen wie bspw. Videocodierung, -bearbeitung, Übersetzer, Stapelverarbeitung.

\newpage
\subsection{Rahmenarchitektur}

\definition[0]{Rahmenarchitektur}{Die Rahmenarchitektur bietet ein (nahezu) vollständiges Programm, das durch Einfüllen geplanter „Lücken” oder Erweiterungspunkten erweitert werden kann. Es enthält die vollständige Anwendungslogik, meistens sogar ein komplettes Hauptprogramm}
\anmerkung[0]{Hollywood-Prinzip}{„Don’t call us – we call you“. Das Hauptprogramm besteht bereits und ruft die Erweiterungen der Benutzer auf.}

\subsection{Dienstorientierte Architektur}

\definition[0]{Dienstorientierte Architektur}{(Service Orientated Architecture, SOA) Die Anwendungen des Programms werden aus unabhängigen Diensten zusammengestellt. Es ist ein abstraktes Konzept einer Softwarearchitektur.}
\\
Die Dienste sind zentrale Elemente des Unternehmens und stellen gekapselte Funktionalität an andere Dienste und Anwendungen bereit.
\begin{itemize}
    \item Einfaches Herauslösen und Ersetzen eines Dienstes zur Laufzeit
    \item Dynamischen Binden wird durch das Dienstverzeichnis (Sammlung aller registrierten Dienste) als zentralen Bestandteil des Dienstmodells ermöglicht
    \item Dienste kapselt geschäftsrelevante Funktionalität
    \item Programmiersprachen- und Plattfrom-unabhängige Bereitstellung der Dienste und Dienstkompositionen
\end{itemize}

\subsection{Programmfamilie}
\definition[0]{Programmfamilie}{Eine Programmfamilie oder Software-Produktlinie ist eine Menge von Programmen, die erhebliche Anteile von Anforderungen, Entwurfsbestandteilen oder Softwarekomponenten gemeinsam haben}
Ziele:
\begin{itemize}
    \item Ausnutzen von Gemeinsamkeiten
    \item Wiederverwendung von Entwürfen, Spezifikationen und Anforderungen, Softwarekomponenten, Bibliotheken
    \item Reduktion der Entwicklungs- und Wartungskosten
\end{itemize}

\subsection{Abstrakte/virtuelle Maschine}
Die Benutztrelation zwischen mehreren abstrakten Maschinen ist hierarchisch, d.h. zyklenfrei.
Eine abstrakte Maschine wird in der Regel von einem oder mehreren Modulen oder Paketen implementiert. Dabei werden die Softwarebefehle und -objekte von den Schnittstellen dieser Module bereitgestellt.
Die darunter liegende Maschine muss ganz oder teilweise verdeckt werden, um Inkonsistenzen der beiden Maschinen zu vermeiden.
Die Befehle der abstrakten Maschine sollen so gewählt werden, dass sie in einer Vielzahl von Programmen verwendet werden können.