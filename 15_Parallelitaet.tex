\section{Parallelität}
\subsection{Die Mooresche Regel (Moore's Law)}
\begin{itemize}
    \item Moore, G.E.:
    “The complexity for minimum component costs has increased at a rate of roughly a factor of two per year.” (1965)
    \item Moore, G.E.:
    “The new slope might approximate a doubling every two years, rather than every year, by the end of the decade.” (1975)
\end{itemize}
\fig{Moore's Law}{./data/moore}


\definition[0]{parallel / nebeläufig}{gleichzeitig ablaufend, simultan}
In der Vorlesung wird nur \important{gemeinsamer Speicher} (shared memory) behandelt.

\subsubsection{Prozess (engl. Process)}
\begin{itemize}
    \item Wird durch \babyblue{Betriebssystem} erzeugt
    \item Enthält Informationen über Programmressourcen und Ausführungszeiten, z.B. 
    \begin{itemize}
        \item Code-Segment (Programminstruktionen)
        \item Daten-Segment (für globale Variablen, Keller, Halden)
        \item Mind. 1 Kontrollfaden
    \end{itemize}
    \item CPU-Kontextwechsel zwischen Prozessen langsam
\end{itemize}

\newpage
\subsubsection{Kontrollfaden}
\begin{itemize}
    \item Instuktionsfolge, die ausgeführt wird
    \item Existiert in einem Prozess
    \item Ein Faden hat eigenen 
    \begin{itemize}
        \item Befehlszeiger
        \item Keller
        \item Register
    \end{itemize}
    \item Teilt sich mit anderen Fäden des gleichen Prozesses
    \begin{itemize}
        \item Adressraum
        \item Code/Daten-Segmente
        \item Andere Ressourcen (z.B. geöffnete Dateien, Sperren, etc.)
    \end{itemize}
    \item CPU-Kontextwechsel zw. Fäden des gleichen Prozesses schneller als zw. unterschiedlichen Prozessen.
\end{itemize}

\fig{Prozesse und Kontrollfäden}{./data/processthread}

\newpage
\section{Parallelität in Java}
    \begin{code}
        \javaPublic \javaKeyword{interface}Runnable \{\\
        \null\quad\javaPublic abstract \javaVoid run();\\
        \}\\
    \end{code}
    \begin{code}
        \javaPublic \javaClass Thread \javaKeyword{implements}Runnable \{\\
        \null\quad\javaPublic Thread(String name);\\
        \null\quad\javaPublic Thread(Runnable target)\\
        \null\quad\javaPublic \javaVoid start();\\
        \null\quad\javaPublic \javaVoid run();\\
        \null\quad\dots\\
        \}\\
    \end{code}
    \important{Threads immer mit \texttt{.start()} starten, nicht \texttt{.run()}!}

\subsection{\texttt{Runnable} vs. \texttt{Thread}}
\begin{itemize}
    \item Die Kapselung der Aufgabe in eine \important{Runnable} erzeugt weniger Overhead
    \item Im Gegensatz zu Thread ist Runnable serialisierbar $\Rightarrow$ kann über das Netzwerk versendet werden ($\Rightarrow$ master/worker).
\end{itemize}

\subsubsection{Koordination}
\begin{itemize}
    \item \babyblue{Wechselseitiger} Ausschluss
    \begin{itemize}
        \item Markierung kritischer Abschnitte, die nur von einer Aktivität gleichzeitig betreten werden dürfen
        \item \javaKeyword{synchronized}
    \end{itemize}
    \item Warten auf \glqq\babyblue{Ereignisse}\grqq\ und Benachrichtigung
    \begin{itemize}
        \item Aktivitäten können auf Zustandsänderungen warten, die durch andere verursacht werden.
        \item Aktivitäten informieren andere, wartende Aktivitäten über Signale
        \item \texttt{Thread\#wait()} | \texttt{Thread\#notify()} | \texttt{Thread\#notifyAll()}
    \end{itemize}
    \item \babyblue{Unterbrechungen}
    \begin{itemize}
        \item Eine Aktivität, die auf ein nicht (mehr) eintretendes Erergnis wartet, kann über eine Ausnahmebedingung abgebrochen werden.
        \item \texttt{Thread\#interrupt()}
    \end{itemize}
\end{itemize}

\newpage

\subsubsection{Kritische Abschnitte}
\begin{itemize}
    \item Ein Bereich, in dem gleichzeitige Zugriffe auf gemeinsam genutzte Daten stattfinden, ist ein \babyblue{kritischer Abschnitt}
    \item Um \babyblue{Wettlaufsitutationen} zu vermeiden, müssen solche kritischen Abschnitte geschützt werden.
    \begin{itemize}
        \item Nur \babyblue{eine} Aktivität darf einen kritischen Abschnitt \babyblue{gleichzeitig} bearbeiten.
        \item \babyblue{Vor dem Betreten} eines kritischen ABschnitts muss sichergestellt sein, dass ihn keine andere aktivität ausführt.
    \end{itemize}
    \item \babyblue{Atomarität}: Java garantiert, das Zugriffe auf einzelne Variablen immer atomar erfolgen.
    \begin{itemize}
        \item Ausgenommen sind 64-Bit-Variablen (\javaKeyword{double} \javaKeyword{long})
    \end{itemize}
    \item Zugriffe auf mehrere Variablen, oder aufeinanderfolgende Lese- und Schreiboperationen werden nicht atomar ausgeführt.
    \begin{itemize}
        \item Vorsicht auch bei \javaKeyword{i++} sieht atomar, ist es aber nicht.
        \item Besteht aus lesen, bearbeiten und schreiben
    \end{itemize}
    \item Auch wenn Zugriff atomar erfolgt, ist dennoch eine Synchronisation erforderlich und im Cache zwischengespeicherte Daten sicher abzugleichen.
\end{itemize}

\fig{Waiting-Area und Monitor}{./data/parallel/monitor_wait}

\vspace{0.5cm}
\definition[0]{Verklemmung}{Eine Verklemmung (engl. Deadlock) ist eine Blockade, die durch eine zyklische Abhängigkeit von Fäden auf Ressourcen hervorgerufen wird. Eine Verklemmung führt dazu, dass alle beteiligten Fäden ewig im Wartezustand verharren}

\newpage

\subsubsection{Semaphoren}
\begin{itemize}
    \item Wird mit einer Anzahl von \glqq Genehmigungen\grqq initialisiert
    \item acquire blockiert, bis eine \glqq Genehmigung\grqq verfügbar ist und erniedrigt anschließend Anzahl der \glqq Genehmigungen\grqq um 1
    \item release erhöht Anzahl der \glqq Genehmigungen\grqq um 1
\end{itemize}
\null
\begin{code}
    \javaPublic \javaClass Semaphore \{\\
        \null\quad\javaPrivate \javaInt count;\\\null\\
        \null\quad    \javaPublic \javaKeyword{synchronized} \javaVoid aquire()  \javaThrows InterruptedException \{\\
        \null\quad\quad        \javaWhile (count <= 0) \{ wait(); \}\\
        \null\quad\quad\quad            -\null-count;\\
        \null\quad    \}\\
        \null\quad\javaPublic \javaKeyword{synchronized} \javaVoid release() \{\\
        \null\quad\quad            ++count;\\
        \null\quad\quad            notifyAll();\\
        \null\quad        \}\\
        \null\quad        \javaPublic Semaphore (\javaInt capacity) \{\\
        \null\quad\quad            count = capacity;\\
        \null\quad        \}\\
    \}
\end{code}

\subsection{Bewertung von parallelen Algorithmen}
\subsubsection{Gesamtlaufzeit}
\[
    \mathcal{T}(p) = \sigma + \frac{\pi}{p}, \text{ wobei}
\]
 $\sigma$ die Zeit für die Ausführung des sequentiellen Teils und\\
 $\pi$ die Zeit für die sequentielle Ausführung des parallelisierbaren Teils darstellt.
 
 \subsubsection{Beschleunigung (\textit{Speedup})}
 \[
    \mathcal{S}(p)=\frac{\mathcal{T}(1)}{\mathcal{T}(p)}
 \]
 \subsubsection{Effizienz}
 \[
    \mathcal{E}(p) = \frac{\mathcal{T}(1)}{p\cdot \mathcal{T}(p)} = \frac{\mathcal{S}(p)}{p}
 \]
 Idealfall:\\
 $\Rightarrow$ $\mathcal{S}(p)=p$ und $\mathcal{E}(p)=1$