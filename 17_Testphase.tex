\section{Testphase}

\subsection{Definitionen}

\subsubsection{Arten von Testhelfern}

\definition[0]{Stummel}{Ein Stummel (engl. stub) ist ein nur rudimentär implementierter Teil der Software und dient als Platzhalter für noch nicht umgesetzte Funktionalität}
\definition[0]{Attrappe}{Eine Attrappe (engl. dummy) simuliert die Implementierung zu Testzwecken}
\definition[1]{Nachahmung}{Eine Nachahmung (engl. mock object) ist eine Attrappe mit zusätzlicher Funktionalität, wie bspw. das Einstellen der Reaktion der Nachahmung auf bestimmte Eingaben oder das Überprüfen des Verhaltens des „Klienten“}

\subsubsection{Arten von Fehlern}

\definition[0]{Anforderungsfehler (Fehler im Pflichtenheft)}{z.B.: Inkorrekte Angaben der Benutzerwünsche, unvollständige Angaben über Anforderungen, Inkonsistenz verschiedener Anforderungen oder Undurchführbarkeit}
\definition[0]{Entwurfsfehler (Fehler in der Spezifikation)}{Zu den Entwurfsfehlern gehören eine unvollständige oder fehlerhafte Umsetzung der Anforderung, Inkonsistenz der Spezifikation oder des Entwurfs, sowie Inkonsistenz zwischen Anforderung, Spezifikation und Entwurf.}
\definition[0]{Implementierungsfehler (Fehler im Programm)}{Als Implementierungsfehler zählen Defekte im Programm, ausgelöst durch die fehlerhafte Umsetzung der Spezifikation}

\subsection{Fehleraufdeckung ist das Ziel der Testverfahren}
\begin{itemize}
    \item Vollständiges Testen aller Kombinationen aller Eingabewerte ist nicht möglich
    \item Korrektheit nur mit formalem Korrektheitsbeweis möglich, welcher nur für kleine Programme möglich ist.
\end{itemize}

\subsection{Arten von Fehlern}
\begin{itemize}
    \item Ein \babyblue{Versagen} oder \babyblue{Ausfall} ist die Abweichung des Verhaltens der Software von der Spezifikation.
    \item Ein \babyblue{Defekt} ist ein Mangel in einem Softwareprodukt, der zu einem Versagen führen kann.
    \item Ein \babyblue{Irrtum} oder \babyblue{Herstellungsfehler} ist eine menschliche Aktion, die einen Defekt verursacht.
\end{itemize}

\subsection{Transformation in Zwischensprache}
\fig{Beispieltransformation}{./data/transformation_zwischensprache}

Einen \babyblue{Grundblock} bezeichnet eine maximal lange Folge fortlaufender Anweisungen der Zwischensprache,
\begin{itemize}
    \item in die der Kontrollfluss nur am Anfang eintritt
    \item die außer am Ende keine Sprungbefehle enthält.
\end{itemize}

\newpage

\subsection{Kontrollflussgraph}
Ein \babyblue{Kontrollflussgraph} eines Programms P ist ein gerichteter Graph G mit 
\[G=(N,E,n_{start},n_{stopp})\]
wobei
\begin{itemize}
    \item N die Menge der Grundblöcke in P,
    \item E $\subset$ N$\times$N die Menge der Kanten, wobei die Kanten die Ausführungsreihenfolge von je zwei Grundblöcken angeben.
    \item $n_{start}$ der Startblock
    \item $n_{stopp}$ der Stoppblock
\end{itemize}
\fig{Beipiel Kontrollflussgraph}{./data/kontrollflussgraph}

\newpage

\fig{Pfadüberdeckungsbeispiel}{./data/pfazeigüberd}
\definition[0]{Die Pfadüberdeckung}{fordert die Ausführung aller unterschiedlichen, vollständigen Pfade im Programm.}
\begin{itemize}
    \item Anweisungsüberdeckung
    $A = \{(Start,1,2,3,4,5,Stopp)\}$
    \item Zeigüberdeckung
    $Z = A \cup \{(Start,1,3,5,Stopp)\}$
    \item Pfadüberdeckung
    $P = Z \cup \{(Start,1,3,4,5,Stopp)\} \cup \{(Start,1,2,3,5, Stopp)\}$
\end{itemize}

\subsection{Äquivalenzklassen}
Aufteilung der einzelnen Parameter in Klassen: Unterscheide zulässige und unzulässige Eingaben und bilde das Kreuzprodukt aus allen Parametern ($\Rightarrow$ Bilde alle Parameterklassenkombinationen).
\\\\
Ein Beispiel:
\\
\begin{code}
    \javaPublic \javaInt example(\javaInt zahl) \{\\
    \quad\dots\ \\\}
\end{code}
\\
Die Methode gebe für positive Werte von $zahl$, die eine ganzzahle Wurzel haben, eben diese aus. Für negative Werte folgt eine IllegalArgumentException. Ist die Wurzel zur Zahl nicht ganzzahlig, wird -1 ausgegeben.
\\
\\
Es folgen die Äquivalenzklassen:
\begin{itemize}
    \item Zahl mit einer ganzzahligen Wurzel
    \item Zahl ohne ganzzahlige Wurzel
    \item Negative Zahl
\end{itemize}

\subsection{Testwerkzeuge}
\subsubsection{Zusicherungen (engl. assertions)}
\begin{itemize}
    \item Boolesche Funktion die vor und Nachbedingung überprüft.
    \item Invarianten in Datenstruktur.
    \item Werden zur Laufzeit überprüft
    \begin{itemize}
        \item Muss in Java explizit aktiviert werden
        \item Im Fehlerfall melden sie sich mit einer Ausnahme oder Fehlermeldung
    \end{itemize}
\end{itemize}

\subsubsection{Zusicherungen in Java}
\begin{code}
    assert Zusicherung;
\end{code}

Falls Zusicherung false ergibt, wird Ausnahme AssertionError ausgelöst.

\subsubsection{Benutzung von Zusicherungen | In Java}
\begin{itemize}
    \item Konvention für \babyblue{öffentliche} Methoden
\begin{itemize}
    \item Überprüfung der Eingabeparameter nicht mit zusicherungen, sondern mit IllegalArgumentException.
    \item \babyblue{Folge}: Klientprogramme können darauf reagieren
\end{itemize}
\item Konvention für \babyblue{private} Methoden
\begin{itemize}
    \item Eingabeparameter, Nachbedingungen und Invarianten aller privater Methoden mit Zusicherungen überprüfen.
    \item Grund: Verletzung ist unerwarteter Defekt
    \item Aber: Falsche Parameter bei öffentlichen Methoden sind nicht unerwartet.
\end{itemize}
\item Zusicherungen können Missverständnisse und Fehlinterpretationen der Programmiererer aufdecken.
\item In Produktionsumgebungen werden die Zusicherungen abgeschaltet.
\item Beim Auftreten eines Defekts oder beim Testen werden diese wieder zugeschaltet.
\item \textbf{Zusicherungen sind keine Testfälle}
\begin{itemize}
    \item Es werden lediglich bestimmte Bedingungen im Programmlauf überprüft.
\end{itemize}
\end{itemize}

