\subsection{Abnahmephase}
\definition[0]{Abnahmephase}{Das Produkt wird vom Auftraggeber abgenommen und in Betrieb genommen. Nun unterliegt das Produkt Wartung und Pflege}
\\
Tätigkeiten in der Abnahmephase:
\begin{itemize}
    \item Übergeben des Produkts inkl. der gesamten Dokumentation
    \item Abnahmetest
    \item Abnahmeprotokoll
\end{itemize}
\\
Die eigentliche Abnahme erfolgt nach erfolgreichem Abnahmetest mit der schriftlichen Abnahme durch den Auftraggeber.
\\
Man muss zusätzlichen den Unterschied zwischen Selbstwartung und Wartung/Pflege durch den Arbeitgeber unterscheiden. Macht der Auftraggeber dies, so benötigt er:
\begin{itemize}
    \item Die gesamte Entwurfs- und Implementierungsdokumentation
    \item Einfürung in die Architektur
    \item Alle Testfälle und die gesamten Testeinrichtungen
\end{itemize}

\subsection{Einführungsphase}
Tätigkeitend er Einführungsphase:
\begin{itemize}
    \item Produkt installieren
    \item Mitarbeiter schulen
    \item Inbetriebnahme
\end{itemize}

\\

\subsubsection{Direkte Umstellung}
Direkte Installation des neuen Systems.
\\
Günstig, aber risikoreich

\subsubsection{Parallellauf}
Das neue System zunächst parallel zum alten laufen lassen.
\\\\
Vorteile:
\begin{itemize}
    \item Ergebnisse mit denen des alten überprüfen
    \item Sicherheit bei Nicht-Funktionieren
\end{itemize}
\\\\
Nachteile:
\begin{itemize}
    \item Ressourcenaufwändig, da Zeit und Mitarbeiter für den Lauf bereitgestellt werden müssen
    \item Probleme durch zwei parallele Systeme
\end{itemize}

\subsubsection{Versuchslauf}
Verwende das neue System zunächst in einem Probelauf mit historischen Daten oder schrittweise Einführung.
\\
Auch dabei.