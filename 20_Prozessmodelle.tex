\section{Prozessmodelle}

\subsection{Programmieren durch Probieren}
\begin{itemize}
    \item Auch \glqq code \& fix\grqq\ oder \glqq trial \& error\grqq\  genannt
    \item Vorgehen
    \begin{itemize}
        \item Vorläufiges Programm erstellen
        \item Anforderung, Entwurf, Testen, Wartung überlegen
        \item Programm entsprechend \glqq verbessern\grqq
    \end{itemize}
    \item Eigenschaften
    \begin{itemize}
        \item \glqq nutzlosen\grqq\ Zusatzaufwand
        \item Erzeugt schlecht stukturierten Code wegen unsystematischer Verbesserungen und fehlender Entwurfsphase
    \end{itemize}
    \item Probleme
    \begin{itemize}
        \item Mangelhafte Aufgabenerfüllung wegen Fehlens der Anforderungsanylyse
        \item Wartung/Pflege kostspielig, da Programm nicht darauf vorbereitet
        \item Dokumentation nicht vorhanden
        \item Für Teamarbeit vollständig ungeeignet, da keine Aufgabenaufteilung vorgesehen.
    \end{itemize}
\end{itemize}

\subsection{Wasserfallmodell (auch Phasenmodell)}
\begin{itemize}
    \item Unterteilt in Phasen. Siehe oben.
    \item Jede Aktivität wird streng sequentiell vollständig ausgeführt.
    \item Am Ende jeder Aktivität steht ein fertiges Dokument
    \item Benutzerbeteiligung nur in der Definitionsphase vorgesehen.
    \item Probleme
    \begin{itemize}
        \item Keine phasenübergreifende Rückkopplung vorgesehen
        \item Parallelisierungspotential möglicherweise nicht richtig ausgeschöpft
        \item Zwingt zu genauen Spezifikation auch schlecht verstandener Benutzerschnittstellen
    \end{itemize}
\end{itemize}

\newpage

\subsection{V-Modell 97}
\begin{itemize}
    \item V wie Vorgehensmodell
    \item Jede Aktivität hat einen eigenen Prüfungsschritt
\end{itemize}
Die Phasen sind im V-Modell-XT nicht einer Reihenfolge zugeordnet.


\fig{V-Modell 97}{./data/vModel97}
\fig{V-Modell Produktzustände}{./data/vModelxt}


\subsection{Prototypmodell}
Erstelle zunächst einen Prototyp mit mindester Funktionalität. Dieser wird dem Kunden vorgezeigt und Verbesserungen sowie Anforderung gesammelt.
\important{Danach wird der Prototyp weggeworfen} und der Prozess mit einem anderen Modell (Wasserfall, V-Modell) fortgeführt.
\\
\fig{Prototypmodell}{./data/prototyp}


\subsection{Iteratives Modell}
\begin{itemize}
    \item Idee: Zumindest Teile der Funktionalität lassen sich klar definieren und realisieren
    \item Daher: Funktionalität wird Schritt für Schritt erstellt und dem Produkt \glqq hinzugefügt\grqq
    \item Gleiche Vorteile und Einsatzgebiete wie Prototypmodell
\end{itemize}
Unterschiedliche Ansätze für Planungs- und Analysephase:
\begin{itemize}
    \item \important{Evolutionär}: Plane und analysiere nur den Teil, der als nächstes hinzugefügt wird
    \item \important{Inkrementell}: Plane und analysiere alles und iteriere dann n-mal über Entwurfs-, Implmentierungs- und Testphase
    \item[$\rightarrow$] Mischformen und Flexibilität angebracht
\end{itemize}
\fig{Iteratives Modell}{./data/iterativ}

\subsection{Synchronisiere und Stabilisiere}
\begin{itemize}
    \item Auch \glqq Microsoft Modell\grqq
    \item Ansatz
    \begin{itemize}
        \item Organisiere die 200 Programmierer eines Projektes in \glqq kleinen Hacker-Teams \grqq
        \item Aber: Synchronisiere regelmäßig
        \item Und Stabilisiere regelmäßig
    \end{itemize}
    \item Drei Phasen
    \begin{itemize}
        \item Planungsphase
        \item Entwicklungsphase in 3 Subprojekten (Milestones)
        \item Stabilisierungsphase
    \end{itemize}
\end{itemize}
\begin{itemize}
    \item Planungsphase
    \begin{itemize}
        \item 3-12 Monate
    \end{itemize}
    \item Developement
    \begin{itemize}
        \item 6-12 Monate
        \item Wichtigkeit unterteilt
        \item Unterteilt in 3 Milestones
    \end{itemize}
    \item Stabilisiere
    \begin{itemize}
        \item 3-8 Monate
    \end{itemize}
\end{itemize}
\begin{itemize}
    \item Pro
    \begin{itemize}
        \item Effektiv durch kurze Produktzyklen
        \item Priorisierung nach Funktion
        \item Fortschritt ohne vollständige Spezifikation möglich
        \item Viele Entwickler arbeiten effektiv in kleinen Gruppen.
    \end{itemize}
    \item Kontra
    \begin{itemize}
        \item Ungeeignet für manche Art von Software
        \item Mündliche Arbeitsweise 
        \item Alle 18 Monate sind 50\% des Codes überarbeitet worden
    \end{itemize}
\end{itemize}


\subsection{Extremes Programmieren - XP}
\subsubsection{Paarprogrammierung}
Zwei Programmierer arbeiten an einer Tastatur und Maus, wobei die Qualität des Codes gesteigert werden soll. Dies verdoppelt jedoch nahezu die Kosten. Es kompensiert jedoch fehlende Inspektionen/Reviews.
\begin{itemize}
    \item Fahrer: Denkt an Implementierung des Algorithmus
    \item Beifahrer: Denkt strategisch und führt ständige Durchsicht durch
\end{itemize}

\subsubsection{Testgetriebene Entwicklung [TDD]}
Motiviere jede Verhaltensänderung am Quelltext durch einen automatisierten Test.
Testcode vor Anwendungscode schreiben.
Inkrementeller Entwurf (nur so viel, wie gebraucht wird. Kein vorausschauender Entwurf)

Kleine Schritte

\fig{Test-Driven-Development}{./data/tdd}

\subsubsection{Nachteile von XP}
\begin{itemize}
    \item Kunde muss mitarbeiten und dies auch wollen
    \item Fehlende Produktdokumentation
    \item Nicht reproduzierbarer Prozess
\end{itemize}

\subsection{Scrum}
Drei Säulen der empirischen Verbesserung:
\begin{itemize}
    \item Transparenz
    \item Überprüfung
    \item Anpassung
\end{itemize}

Bei Scrum gibt es folgende Rollen:
\begin{itemize}
    \item Produktverantwortliche (Product Owner) ist für die Wertsteigerung des Produktes und die Anforderungsliste. Legt Anforderungen fest und priorisiert diese (für jeden Sprint)
    \item Scrum Master stellt die Einhaltung der Scrum-Werte und -Techniken sicher. Sorgt für die Produktivität und Funktionalität des Entwicklungsteams. Beseitigt Hindernisse und unterstützt die Zusammenarbeit aller Rollen
    \item Entwickler sind für die Umsetzung der in der Aufgabenliste definierten Aufgaben zuständig
\end{itemize}

In Scrum wird die (veränderliche) Anforderungsliste in 2-4-wöchigen Sprints abgearbeitet
\begin{itemize}
    \item Sprintplanung (\textit{sprint planning})
    \item Tägliches Scrum-Treffen (\textit{daily scrum})
    \item Review-treffen am Ende eines Sprints (\textit{sprint review})
    \item Retrospektive (\textit{sprint retrospective})
\end{itemize}
\fig{Scrum}{./data/scrum}
