\section{Prozessmodelle}

\subsection{Programmieren durch Probieren}
\begin{itemize}
    \item Auch \glqq code \& fix\grqq\ oder \glqq trial \& error\grqq\  genannt
    \item Vorgehen
    \begin{itemize}
        \item Vorläufiges Programm erstellen
        \item Anforderung, Entwurf, Testen, Wartung überlegen
        \item Programm entsprechend \glqq verbessern\grqq
    \end{itemize}
    \item Eigenschaften
    \begin{itemize}
        \item \glqq nutzlosen\grqq\ Zusatzaufwand
        \item Erzeugt schlecht stukturierten Code wegen unsystematischer Verbesserungen und fehlender Entwurfsphase
    \end{itemize}
    \item Probleme
    \begin{itemize}
        \item Mangelhafte Aufgabenerfüllung wegen Fehlens der Anforderungsanylyse
        \item Wartung/Pflege kostspielig, da Programm nicht darauf vorbereitet
        \item Dokumentation nicht vorhanden
        \item Für Teamarbeit vollständig ungeeignet, da keine Aufgabenaufteilung vorgesehen.
    \end{itemize}
\end{itemize}

\subsection{Wasserfallmodell (auch Phasenmodell)}
\begin{itemize}
    \item Unterteilt in Phasen. Siehe oben.
    \item Jede Aktivität wird streng sequentiell vollständig ausgeführt.
    \item Am Ende jeder Aktivität steht ein fertiges Dokument
    \item Benutzerbeteiligung nur in der Definitionsphase vorgesehen.
    \item Probleme
    \begin{itemize}
        \item Keine phasenübergreifende Rückkopplung vorgesehen
        \item Parallelisierungspotential möglicherweise nicht richtig ausgeschöpft
        \item Zwingt zu genauen Spezifikation auch schlecht verstandener Benutzerschnittstellen
    \end{itemize}
\end{itemize}

\newpage

\subsection{\glqq V-Modell 97 \grqq - das \glqq handelsübliche \grqq}
\begin{itemize}
    \item V wie Vorgehensmodell
    \item Jede Aktivität hat einen eigenen Prüfungsschritt
\end{itemize}
\fig{V-Modell 97}{./data/vModel97}