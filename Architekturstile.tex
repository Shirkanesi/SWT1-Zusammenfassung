\section{Architekturstile}
\definition[0]{Architekturstil}{Die Architekturstile legen den \important{Grobaufbau} eines Softwaresystems fest.}

\definition[0]{Schichtenarchitektur}{ist die Gliederung einer Softwarearchitektur in hierarchisch geordnete Schichten. Eine Schicht besteht aus einer Menge von Software- Komponenten (Module, Klassen, Objekte, Pakete) mit einer wohldefinierten Schnittstelle, nutzt die darunter liegenden Schichten und stellt seine Dienste darüber liegenden Schichten zur Verfügung.}
\anmerkung[1]{}{Zwischen den einzelnen Schichten ist die Benutztrelation linear, baumartig, oder ein azyklischer Graph. Innerhalb einer Schicht ist die Benutztrelation beliebig}
\fig{Beispiel für eine Drei-Schichten-Architektur}{./data/architekturstile/3_schichten_architektur.png}

\begin{itemize}
    \item Eine Schicht (engl. \textit{layer, tier}) ist ein Subsystem, welches Dienste für andere Schichten zur Verfügung stellt, mit folgenden Einschränkungen:
    \begin{itemize}
        \item Eine Schicht nutzt nur Dienste von niedrigeren Schichten
        \item Eine Schicht nutzt keine höheren Schichten
    \end{itemize}
    \item Eine Schicht kann horizontal in mehrere, unabhängige Subsysteme, auch Partitionen genannt, aufgeteilt werden
    \begin{itemize}
        \item Partitionen bieten Dienste für andere Partitionen der gleichen Schicht an
    \end{itemize}
\end{itemize}
\anmerkung[1]{Transparente und Intransparente Schichten}{Eine Schicht in einer transparenten Schicht kann auf alle Beliebigen Schichten im Stapel zugreifen. Schichten im intransparenten Modell können Dies nicht.}

\definition[0]{Klient/Dienstgeber}{}
\definition[0]{Partnernetze}{}
\definition[0]{Datenablage}{}
\definition[0]{Modell-Präsentation-Steuerung}{}
\definition[0]{Fließband}{}
\definition[0]{Rahmenarchitektur}{}
\definition[0]{Dienstorientierte Architektur}{}

\subsection{Abstrakte/virtuelle Maschine}
Die Benutztrelation zwischen mehreren abstrakten Maschinen ist hierarchisch, d.h. zyklenfrei.
Eine abstrakte Maschine wird in der Regel von einem oder mehreren Modulen oder Paketen implementiert. Dabei werden die Softwarebefehle und -objekte von den Schnittstellen dieser Module bereitgestellt.
Die darunter liegende Maschine muss ganz oder teilweise verdeckt werden, um Inkonsistenzen der beiden Maschinen zu vermeiden.
Die Befehle der abstrakten Maschine sollen so gewählt werden, dass sie in einer Vielzahl von Programmen verwendet werden können.