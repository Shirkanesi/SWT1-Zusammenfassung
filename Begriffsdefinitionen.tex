\section{Definitionen}
% Informatik
\definition{Informatik}{ist die Wissenschaft von den (natürlichen und künstlichen) Informationsprozessen.}
% Softaretechnik
\definition{Softwaretechnik}{(engl. Software engeneering) ist die technologische und organisatorische Disziplin zur sytematischen Entwicklungund Pflege von Softwaresystemen, die spezifizierte funktionale und nicht-funktionale Attribute erfüllen.}
% Softwarekonfigurationsverwaltung
\definition{Softwarekonfigurationsverwaltung}{(software configuration management) ist die Disziplin zur Verfolgung und Steuerung der Evolution von Software.}
% Softwarekonfiguration
\definition{Softwarekonfiguration}{ist eine \important{eindeutig} benannte Menge von Software-Elementen mit den jeweils gültigen Versionsangaben, die zu einem bestimmten Zeitpunkt im Produktlebenszyklus in ihrer Wirkungsweise und ihren Schnittstellen aufeinander abgestimmt sind und gemeinsam eine vorgesehene AUfgabe erfüllen sollen.}
% Software
\definition{Software}{(engl., "Weichware", Programmatur, im Gegensatz zur apparatur) ist eine Sammelbezeichnung für Programme und Daten, die für den Betrieb von Rechensystemen zur Verfügung stehen, einschl. der zugehörigen Dokumentation.}
% Softwareprodukt
\definition{Softwareprodukt}{ ist ein Produkt für ein in sich abgeschlossenes, für einen Auftraggeber oder einen anonymen Markt bestimmtes Ergebnis eines erfolgreichen durchgeführten Projekts oder Herstellungsprozesses.}
% Teilprodukt
\definition{Teilprodukt}{ beschreibt einen abgeschlossenen Teil eines Produkts.}
