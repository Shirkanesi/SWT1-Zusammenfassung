\section{Definitionen}
% Informatik
\definition{Informatik}{ist die Wissenschaft von den (natürlichen und künstlichen) Informationsprozessen.}
% Softaretechnik
\definition{Softwaretechnik}{(engl. Software engeneering) ist die technologische und organisatorische Disziplin zur sytematischen Entwicklung und Pflege von Softwaresystemen, die spezifizierte funktionale und nicht-funktionale Attribute erfüllen.}
% Softwarekonfigurationsverwaltung
\definition{Softwarekonfigurationsverwaltung}{(software configuration management) ist die Disziplin zur Verfolgung und Steuerung der Evolution von Software.}
% Softwarekonfiguration
\definition{Softwarekonfiguration}{ist eine \important{eindeutig} benannte Menge von Software-Elementen mit den jeweils gültigen Versionsangaben, die zu einem bestimmten Zeitpunkt im Produktlebenszyklus in ihrer Wirkungsweise und ihren Schnittstellen aufeinander abgestimmt sind und gemeinsam eine vorgesehene Aufgabe erfüllen sollen.}
% Software
\definition{Software}{(engl., "Weichware", Programmatur, im Gegensatz zur apparatur) ist eine Sammelbezeichnung für Programme und Daten, die für den Betrieb von Rechensystemen zur Verfügung stehen, einschl. der zugehörigen Dokumentation.}
% Softwareprodukt
\definition{Softwareprodukt}{ ist ein Produkt für ein in sich abgeschlossenes, für einen Auftraggeber oder einen anonymen Markt bestimmtes Ergebnis eines erfolgreichen durchgeführten Projekts oder Herstellungsprozesses.}
% Teilprodukt
\definition{Teilprodukt}{ beschreibt einen abgeschlossenen Teil eines Produkts.}

\definition{Softwarearchitektur}{Gliederung eines Softwaresystems in Komponenten und Subsysteme, Spezifikation der Komponenten und Subsysteme, Aufstellen der \anfuehrung{Benutzt}-Relation zwischen Komponenten und Subsystemen (Optional: Feinentwurf und Zuweisung von SW-Komponenten und Subsystem zu HW-Einheiten).}

\definition{Ein Modul}{ist eine Menge von Programm-Elementen, die nach dem Geheimnisprinzip gemeinsam entworfen und geändert werden.}
\definition{Geheimnisprinzip}{Jedes Modul verbirgt eine wichtige Entwurfsentscheidung hinter einer wohldefinierten Schnittstelle, die sich bei einer Änderung der Entscheidung nicht mitändert.}
\definition{Benutztrelation, (benutzt)}{Programmkomponente A benutzt Programmkomponente B genau ann, wenn A für den korrekten Ablauf die Verfügbarkeit einer korrekten Implementierung von B erfordert.}

\definition{Eine abstrakte Maschine oder Virtuelle Maschine}{ist eine Menge von Softwarebefehlen und -objekten, die auf einer darunterliegenden (abstrakten oder realen) Maschine aufbauen und diese ganz oder teilweise verdecken können.}-

\definition{Def. Anwendungsfall}{(engl. use-case): Ein Anwendungsfall ist eine typische, gewollte Interaktion eines oder mehrerer Akteure (engl. actor) mit einem (geschäftlichen oder technischen) System.}