\section{Modularer Entwurf}
\definition{Modul}{Ein Modul ist eine Menge von Programmelementen, die nach dem Geheimnisprinzip gemeinsam entworfen und geändert werden.}
\definition{Geheimnisprinzip / Kapselungsprinzip}{Jedes Modul verbirgt eine wichtige Entwurfsentscheidung hinter einer wohldefinierten Schnittstelle, die sich bei einer Änderung der Entscheidung nicht mit ändert}
\definition{Benutztrelation}{Programmkomponente A benutzt Programmkomponente B $\Leftrightarrow$ A benötigt für den korrekten Ablauf die korrekte Implementierung von B}

\definition{Programmfamilie / Softwareproduktlinie}{ist eine Menge von Programmen, die erhebliche Anteile von Anforderungen, Entwurfsbestandteilen oder Softwarekomponenten gemeinsam haben.}

\definition[0]{Schichtenarchitektur}{ist die Gliederung einer Softwarearchitektur  in hierarchisch geordnete Schichten. Zwischen den einzelnen Schichten ist die Benutztrelation linear, baumartig, oder ein azyklischer Graph. Innerhalb einer Schicht ist die Benutztrelation beliebig.}
\definition{Schicht}{besteht aus einer Menge von Softwarekomponenten (Module, Klassen, Objekte, Pakete) mit einer wohldefinierten Schnittstelle, nutzt die darunter liegenden Schichten und stellt seine Dienste darüber liegenden Schichten zur Verfügung.}
