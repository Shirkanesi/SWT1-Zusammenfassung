\section{Planungsphase}
\definition{Definition}{Die Planungsphase hat das Ziel, in einem Lastenheft das zu 
entwickelnde System in Worten des Kunden zu beschreiben und die 
Durchführbarkeit des Projektes zu überprüfen.}
\\
Folgende Bestandteile fallen in die Planungsphase:
\begin{itemize}
    \item Durchfürbarkeitsstudie
    \item Lastenheft
    \item Projektkalkulation
    \item Projektplan
\end{itemize}

\definition{Szenario}{Beschreibung eines Ereignisses odfer einer Folge von Aktionen und Ereignissen. Es beschriebt die Verwendung eines Systems in Textform aus Sicht eines Benutzer. Sie können aber auch in Testphase und Auslieferung eingesetzt werden.}

\subsection{Lastenheft}
\definition{Definition}{Das Lastenheft beschreibt das System in der Sprache des Kunden. Insbesondere soll also der Kunde einen Überblick erhalten.}

\subsubsection{Bestandteile}
\begin{itemize}
    \item Zielbestimmung
    \item Produkteinsatz/Zweck/Zielgruppe/Plattform
    \item Funktionale Anforderungen
    \item Produktdaten (welche Daten werden gespeichert?)
    \item Nichtfunktionale Anforderungen
    \item Systemmodelle (Szenarien, Anwendungsfälle)
    \item Glossar (Lexikon zu Produktbeschreibungen)
\end{itemize}

\subsubsection{Anforderungen}
\definition{Funktionale Anforderungen}{Beschreiben das Verhalten, die Reaktionen des Systems auf Eingaben und Daten, oder die Interaktion zwischen dem System und der Systemumgebung unabhängig von der Implementierung. Sie werden als Aktionen formuliert.}

\definition{Nichtfunktionale Attribute}{Beschreiben die Eigenschaften des Systems oder der Domäne und werden als Einschränkungen bzw. Zusicherungen formuliert.}

\definition{Qualitätsanforderungen}{Beschreiben die verlangte Qualität der Funktionen wie bspw. Antwortzeiten.}
\\
\\
Eine Liste von einigen Qualitätsanforderungen:
\begin{itemize}
    \item Benutzbarkeit
    \item Zuverlässigkeit
    \item Robustheit
    \item Sicherheit
    \item Leistungsfähigkeit
    \item Skalierbarkeit
    \item Verfügbarkeit
    \item Wartbarkeit
    \item Portierbarkeit
\end{itemize}

\definition{Einschränkungen}{Werden vom Kunden vorgegeben und behandelt möglicherweise auch technische Vorgaben ("Code muss in Java geschrieben werden").}


\subsection{Durchführbarkeitsuntersuchung}
\begin{itemize}
    \item Fachliche Durchführbarkeit
    \item Alternative Lösungsvorschläge?
    \item Personelle Durchführbarkeit
    \item Prüfung der Risiken
    \item Ökonomische Durchführbarkeit (Aufwands- und Terminschätzung und Wirtschaftslichkeitsrechnung)
    \item Rechtliche Gesichtspunkte (Datenschutz, Zertifizierung, Relevante Standards)
\end{itemize}
