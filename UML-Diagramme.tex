\section{UML-Diagramme}

\subsection{Anwendungsfalldiagramm}
\begin{itemize}
    \item Zur \babyblue{Anforderungsspezifikation} – was will der Benutzer von seinem System?
    \item Modellieren typischer \babyblue{Interaktionen} des Benutzers mit dem System
    \item Ermöglicht Kontrolle, ob das System das vom Auftraggeber gewünschte leistet (Design und Implementierung)
\end{itemize}

\definition{Def. Anwendungsfall}{Def. Anwendungsfall (engl. use-case): Ein Anwendungsfall ist eine typische, gewollte Interaktion eines oder mehrerer Akteure (engl. actor) mit einem (geschäftlichen oder technischen) System.}

\fig{Beispiel „Groupware-System“}{./data/Anwendungsfalldiagramm.png}

\section{Aktivitätsdiagramm}
\begin{itemize}
    \item beschreibt einen Ablauf
    \item bestehen aus
    \begin{itemize}
        \item Aktions-, Objekt- und Kontrollknoten, sowie
        \item Objekt- und Kontrollflüssen
    \end{itemize}
\end{itemize}

\fig{Beispiel Aktivitätsfalldiagramm}{./data/uml/Aktivitätsdiagramm.png}
\fig{Elemente eines Aktivitätsdiagramms}{./data/uml/elemente.png}   %Eventuell richtig machen idk lol.
\fig{Semantisches Beispiel}{./data/uml/aktivität_beispiel.png}