\section{UML-Diagramme}

\subsection{Anwendungsfalldiagramm}
\begin{itemize}
    \item Zur \babyblue{Anforderungsspezifikation} – was will der Benutzer von seinem System?
    \item Modellieren typischer \babyblue{Interaktionen} des Benutzers mit dem System
    \item Ermöglicht Kontrolle, ob das System das vom Auftraggeber gewünschte leistet (Design und Implementierung)
\end{itemize}

\definition{Def. Anwendungsfall}{Def. Anwendungsfall (engl. use-case): Ein Anwendungsfall ist eine typische, gewollte Interaktion eines oder mehrerer Akteure (engl. actor) mit einem (geschäftlichen oder technischen) System.}

\fig{Beispiel „Groupware-System“}{./data/Anwendungsfalldiagramm.png}

\section{Aktivitätsdiagramm}
\begin{itemize}
    \item beschreibt einen Ablauf
    \item bestehen aus
    \begin{itemize}
        \item Aktions-, Objekt- und Kontrollknoten, sowie
        \item Objekt- und Kontrollflüssen
    \end{itemize}
\end{itemize}

\fig{Beispiel Aktivitätsfalldiagramm}{./data/uml/Aktivitätsdiagramm.png}
\fig{Elemente eines Aktivitätsdiagramms}{./data/uml/elemente.png}   %Eventuell richtig machen idk lol.
\fig{Semantisches Beispiel}{./data/uml/aktivität_beispiel.png}
\fig{Synchronisieren}{./data/uml/aktivität_sync.png}

\section{Interaktionsdiagramme}

\begin{itemize}
    \item Zeigen die für einen bestimmten Zweck notwendigen Interaktionen zwischen Objekten
    \item Das Klassendiagramm ist Grundlage der Interaktionsdiagramme
    \item Vier Typen
    \begin{itemize}
        \item Kollaborationsdiagramm / Kommunikationsdiagramm
        \begin{itemize}
            \item Schwerpunkt: Struktur der Interaktionspartner
        \end{itemize}
        \item Zeitdiagramm
        \begin{itemize}
            \item Schwerpunkt: Zeitliche Koordination
        \end{itemize}
        \item Interaktionsübersicht
        \begin{itemize}
            \item Aktivitätsdiagramm zur Veranschaulichung komplexer Sequenzdiagramme
        \end{itemize}
        \item Sequenzdiagramm
        \begin{itemize}
            \item Schwerpunkt: Nachrichtenaustausch
        \end{itemize}
    \end{itemize}
\end{itemize}

\subsection{Aktivitätsdiagramm}
\fig{Aktivitätsdiagramm: Beispiel}{./data/uml/sequenz.png}


\begin{table}[h]
\begin{tabular}{l|l|l}
Operator & Bed./Parameter                         & Bedeutung                                                                                \\
\hline
alt      & {[}bed.1{]},{[}bed.1{]},...,{[}else{]} & Nur eine der Alternativen wird ausgeführt                                                \\
break    & {[}bed{]}                              & Ist die Bed. wahr, dann wir nur der Block ausgeführt und anschließend\\&& endet das Szenario. \\
opt      & {[}bed{]}                              & Optionale Sequenz. Wird nur Ausgeführt, wenn Bed. wahr ist.                              \\
par      &                                        & Enthaltene Teilsequenzen werden parallel ausgeführt.                                     \\
loop     & {[}bed{]}                              & Solange die Bed. wahr ist, wird der Block ausgeführt.                                   
\end{tabular}
\caption{Operatoren zur Ablaufsteuerung}
\label{tab:ablaufsteuerung}
\end{table}