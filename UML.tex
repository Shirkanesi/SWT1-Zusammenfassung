\section{UML (Universal Modeling Language)}
\definition{Kardinalität}{Die Anzahl der Elemente einer Menge (Ganzzahl $\geq 0$).}
\definition[0]{Multiplizizät}{ein geschlossenes Intervall der zulässigen Kardinalitäten.}
%
\begin{itemize}
    \item 0..1 $\rightarrow$ 0 oder 1
    \item 0.. $\rightarrow$ mindestens 0
    \item x..* $\rightarrow$ mindestens x
    \item x $\rightarrow$ \textbf{genau} x
    \item * $\rightarrow$ 0,1 oder mehrere
\end{itemize}
\important{Keine Angabe heißt nicht Spezifiziert $\rightarrow$ IMMER GENAU 1}

\fig{Sonderfälle}{data/uml/sonderfaelle}


\definition{Qualifizierer}{(eng. qualifier): Ein(e) Attribut(kombination), die eine Partionierung auf der Menge der \important{assoziierten} Exemplaren definiert.}
\definition{Qualifizierte Assoziation}{(engl. qualified association): Eine Assoziation, bei der die Menge der refernzierten Objekte durch einen Qualifizierer partitioniert ist.}


\todo
Static --> unterstrichen
Constructor: <<create>>, name irrelevant.
Abstrakte Methode: kursiv oder \{abstract\} als Suffix
<<interface>> (wird mit einem gestrichekten Pfeil mit weißer Spitze implementiert <<realize>>)
A <<uses>> B :  A benutzt etwas aus B


\subsection{Vererbung}
% Vererbung
\definition[0]{Vererbung}{Es seien A und B Klassen, sowie $\Omega_A$ und $\Omega_B$ die Menge der Objekte, die die Klassen A und B ausmachen. Dann ist B eine \important{Unterklasse/Spezifizierung} von A (oder A eine \important{Oberklasse/Generalisierung} von B), wenn gilt: $\Omega_B\subseteq\Omega_A$.
\newline\indent
Man sagt dann auch, dass B von A erbt. Da jedes Exemplar von B auch ein Exemplar von A ist, heißt die Beziehung zwischen A und B die \anfuehrung{\important{ist-ein-Beziehung}} (engl. is-a relation). Wenn A mehrere Unterklassen hat, so sollten diese Unterklassen in der Regel disjukt sein.
}
\anmerkung[1]{$\Omega_B\subseteq\Omega_A$}{Diese Teilmengenrelation gilt, da $\Omega_A$ alle Klassen enthält, die mindestens die angegebenen Attribute beinhalten, allerdings auch beliebig viele andere.}
% Signaturvererbung
\definition{Signaturvererbung}{(engl. signature inheritance) Eine in der Oberklasse definert und evtl. implementierte Methode überträgt nur ihre Signatuir auf die Unterklasse.}
% Implementierungsvererbung
\definition{Implementierungsvererbung}{(engl. implementation inharitance)Eine in der Oberklasse definierte und implementierte Methode überträgt ihre \important{Signatur und ihre Implementierung} auf die Unterklasse.}
% Überschreiben
\definition{Überschreiben}{(engl. override): Eine geerbte Methode unter Beibehaltung der Signatur wird \important{neu implementiert}}
% Abstrakte Methode
\definition{Abstrakte Methode}{(engl. abstract method): Signaturdefinition ohne Angabe einer Implementierung}
% Schnittstelle
\definition[0]{Schnittstelle}{(engl. interface): Definition einer Menge \important{abstrakter} Methoden, die von den Klassen, die sie implementieren, angeboten werden müssen.}
\anmerkung[0]{}{Schnittstelle nicht direkt instanziierbar.}
\anmerkung[0]{}{Nutzende Klasse darf beliebige implementierende Klasse instanziieren.}
\imp[1]{}{Bei Schnittstellen gibt es keine Vererbung}
\definition[1]{}{Wenn eine Schnittstelle B eine Schnittstelle A \important{erweitert}, dann ist die Menge der abstrakten Methoden von A eine Teilmenge der Mengen der abstrakten Methoden von B ($A\subseteq B$).}
\definition[1]{}{Wenn eine Klasse X eine Schnittstelle A \important{implementiert}, dann ist die Menge der abstrakten Methoden von A eine Teilmenge der Methodendefinitionen von X, wobei X zusätzlich jeweils eine Implementierung angeben \underline{darf}.}

\subsection{Liskovsches Substitutionsprinzip}
\todo{Change image}\\
\principle[0]{Liskovsches Substitutionsprinzip}{(engl. Liskov substitution priciple): In einem Programm, in dem U eine Unterklasse von K ist, kann jedes Exemplar der Klasse K durch ein Exemplar von U ersetzet werden, wobei das Programm weiterhin korrekt funktioniert.)}
\anmerkung[0]{}{Das Substitutionsprinzip fordert nur, dass man ein Exemplar der Unterklasse so verwenden kann, als wäre es ein Exemplar der Oberklasse. Es wird nicht gefordert, dass die Signatur gleich bleibt!}
\anmerkung[0]{}{Die Unterklasse hat die gleichen oder schwächere Vorbedingungen als die Oberklasse\\
Für Methoden heißt das, dass bei einer Substitution der Oberklasse durch eine Unterklasse die Vorbedingungen der Unterklassenmethode ebenfalls erfüllt sind.}
\anmerkung[0]{}{Die Unterklasse bietet die gleichen oder stärkeren Nachbedingungen wie die Oberklasse\\
Für Methoden heißt das, dass bei einer Substitution die Ergebnisse der Unterklassenmethode die Bedingungen erfüllen, die an die Oberklassenmethode gestellt werden.}
\anmerkung[1]{}{Unterklassenmethoden dürfen nicht mehr erwarten und weniger liefern.}

\subsection{Varianzen von Parametern}
% Varianz
\definition[1]{Varianz}{(engl. variance): Modifikation der Typen der Parameter einer überschriebenen Methode.}

% Kovarianz
\definition[0]{Kovarianz}{(engl. covariance): Verwendung einer Spezialisierung des Parametertyps in der überschreibenden Methode}
\imp[0]{}{Bei Eingabeparametern ein Problem}
\anmerkung[1]{}{Bei Rückgabewerten unproblematisch}
% Kontravarianz
\definition[0]{Kontravarianz}{(engl. contravariance): Verwendung einer Verallgemeinerung des Parametertyps in der überschreibenden Methode}
\anmerkung[0]{}{Bei Eingabeparametern unproblematisch}
\imp[1]{}{Bei Rückgabewerten ein Problem}
% Invarianz
\definition[0]{Invarianz}{(engl. invariance): keine Modifikation der Typen}
\anmerkung[1]{}{Invarianz ist nie ein Problem}