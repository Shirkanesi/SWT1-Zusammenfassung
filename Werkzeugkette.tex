\definition{Softwarekonfigurationsverwaltung}{ (engl. software configuration management) ist die Disziplin zur Verfolgung und Steuerung der Evolution von Software}
\definition{(Software-) Konfiguration}{ist eine \babyblue{eindeutig bekannte} Menge von \babyblue{Software-Elementen}, mit den jeweils gültigen \babyblue{Versionsangaben}, die zu einem bestimmten Zeitpunkt im Produktlebenszyklus in ihrer Wirkungsweise und ihren Schnittstellen aufeinander abgesetimmt sind und gemeinsam eine vorgesehene Aufgabe erfüllen sollen. }

\definition{Software-Element}{ist jeder identifizierbare Bestandteil eines Produktes oder einer Produktlinie. Ein Software-Element kann eine einzelne Datei sein, oder auch eine Konfiguration.}

\subsection{Software-Element}
\begin{itemize}
    \item Besitzt systemweit eindeutigen \babyblue{Bezeichner}
    \item Änderung am Element erzeugt neuen Bezeichner, um Fehleridentifikation zu vermeiden
    \item Unterschiede \begin{itemize}
        \item Quellelement: manuell erzeugt, z.B. mit Editor
        \item Abgeleitetes Element: automatisch generiert, z.B. durch Übersetzer
    \end{itemize}
\end{itemize}

\subsection{Versionen}
\begin{itemize}
    \item Eine \babyblue{Version} ist die Ausprägung eines Software-Elementes zu einem bestimmten Zeitpunkt.
    \item \babyblue{Revisionen} sind zeitlich nacheinander liegende Versionen (Entwicklungsstände)
    \item \babyblue{Varianten} sind alternative Versionen
\end{itemize}

\subsection{Ein-/Ausbuchen}
\definition{Depot}{Software-Elemente werden in \babyblue{Depots} gespeichert.}

\subsubsection{Ausbuchen}
\begin{itemize}
    \item Holt Kopie aus Depot
    \item Reserviert Kopie für Ausbucher
    \item Kopie darf geändert und wieder \babyblue{Eingebucht} werden.
\end{itemize}
\subsubsection{Einbuchen}
\begin{itemize}
    \item Scheibt Kopie in Depot zurück
    \item Löscht Reservierung
    \item Speichert Autor, Einbuchungszeit und \babyblue{Logbucheintrag}
    \item Eingebuchtes Element ist nicht mehr änderbar. Erst nach erneuter Ausbuchung.
\end{itemize}

\subsubsection{Optimistisches Ausbuchen}
\begin{itemize}
    \item Keine \babyblue{Ausbuchung} vorher nötig. Man kann einfach bearbeiten.
    \item Vorteile: \begin{itemize}
        \item Mehrere Entwickler können gleichzeitig am gleichen Element arbeiten.
        \item Ein Einzelner kann nicht die Arbeit behindern.
    \end{itemize}
    \item Nachteile \begin{itemize}
        \item Es können gleichzeitig Änderungen an der Selben Datei vorgenommen werden.
        \item Aufwand beim zusammenführen (\babyblue{Merge-Konflikte})
    \end{itemize}
\end{itemize}

\subsection{GIT}
\begin{itemize}
    \item git add: working dir. -> staging area
    \item git commit: staging area -> local repo
    \item git push: local repo -> remote repo
    \item git fetch: remote repo -> local repo
    \item git checkout: local repo -> working dir.
    \item git merge: local repo -> working dir.
\end{itemize}
\definition{HEAD-POINTER}{zeigt auf aktuellen Zustand des Arbeitsordners.}

\# Seite 53 Werkzeugkette.




